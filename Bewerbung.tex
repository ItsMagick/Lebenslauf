%! Author = charon
%! Date = 6/20/23
% Document
\parbox[][0.11\textheight][c]{\linewidth}{ % Box to hold your name and CV title; change the fixed height as needed to match the colored box to the right
    \centering % Horizontally center text
    {\sffamily\Huge Bewerbung auf eine Praktikumsstelle für die Bachelorarbeit}
}
\newline

\section[12pt]
\noindent
Sehr geehrte Damen und Herren,\newline\newline
hiermit bewerbe ich mich bei Ihnen als Praktikant für meine Praxis- und Bachelorarbeit im Rahmen meines Studiums im Bereich IT-Security.\newline\newline
Zurzeit studiere ich im 6. Semester Mobile Computing an der Hochschule Hof und arbeite in der Forschungsgruppe System and Network Security
am Institut für Informationssysteme der Hochschule Hof im Bereich Rowhammer unter der Leitung von Herrn Professor Adamsky.
Bereits sehr früh in meinem Studium stellte ich fest, dass ich Fuß in der Welt der IT-Sicherheit fassen möchte und besuchte Webinare zur persönlichen Weiterbildung außerhalb des Studiums.
Ich habe bereits erste Erfahrungen in der Schwachstellenuntersuchung im Modul IT-Sicherheit erlangt, in dem ich eine wissenschaftliche Arbeit über den CVE 2021-44228 Log4Shell verfasst habe.
Gelegentlich versuche ich mich an Capture The Flags auf den Plattformen Tryhackme, Root.me und Hackthebox. Meine Profile hierzu können Sie aus meinem Lebenslauf entnehmen. \newline\newline
Im Verlauf meines Studiums kam ich mit einigen Technologien, wie Javascript-basierte Frameworks, Java, Flutter, Kotlin, Swift, PHP, C, Assembler,
Reverse-engineering mit IDA, Python, C\#, SQL und NoSQL in Kontakt.
Projekte, die mir besonders Spaß gemacht haben und ebenfalls mein fundamentales Verständnis für die Informatik gestärkt haben, waren die Untersuchung der bereits genannten Schwachstelle
Log4Shell, Reverse-engineering von rudimentären C-Programmen und Hardwarenahe Programmierung eines ATMega32 mit C. Ebenfalls nennenswert ist die Programmierung eines neuronalen Netzes mit Python
zum Analysieren von Aktienkursen. \newline\newline
Außerdem verwende ich für meine Projekte Linux. Aktuell verwende ich die Distributionen Debian, Arch, ParrotOS im Alltag und Selinux für Testzwecke zum Ausprobieren von Hardeningtechniken
an einem Raspberry Pi3. Weitere Erfahrungen habe ich mit den Distributionen OpenSUSE und Ubuntu gesammelt. \newline\newline
Zu meinen fachlichen Kenntnissen kommen noch die Sprachen Deutsch, Englisch, Tschechisch und Französisch dazu. Ich bin Bilingual aufgewachsen, da meine Mutter tschechischer Abstammung ist.
Meine vielseitigen Sprachkenntnisse bewiesen sich bereits oft in internationalen Projekten, wie der Exkursion zu unserer Partneruniversität PSU Abbington in Philadelphia oder
eines Hackathons in Zusammenarbeit mit der Universität Epitech aus Frankreich, als sehr hilfreich. \newline\newline
Zudem bin ich auch ein großer Teamplayer und genieße die Arbeit in Projektteams sehr. Das Manifestiert sich auch in meiner Freizeit. Ich bin Tischtennisspieler im Verein TTC 1990 Hof
und habe dort auch bis vor kurzem als Mannschaftsführer fungiert. Dazu kommt meine hohe Geduld und das Vermögen in großen Stressituaionen die Ruhe zu bewahren. Diese Qualität äußert sich
in meiner Tätigkeit als Jugendtrainer für Tischtennis im Landkreis Hof. \newline\newline
Für weitere Fragen sehen Sie weitere Informationen in meinem Lebenslauf. Ich bin jederzeit unter der Rufnummer und Email auf meinem Lebenslauf erreichbar. Ich würde mich über ein
persönliches Gespräch mit Ihnen sehr freuen und freue mich über Ihre Antwort. \newline\newline
\noindent
Mit freundlichen Grüßen,\newline
Sebastian Peschke